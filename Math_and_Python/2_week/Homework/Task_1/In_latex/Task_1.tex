
% Default to the notebook output style




% Inherit from the specified cell style.





\documentclass[11pt]{article}



\usepackage[T1]{fontenc}
% Nicer default font (+ math font) than Computer Modern for most use cases
\usepackage{mathpazo}

% Basic figure setup, for now with no caption control since it's done
% automatically by Pandoc (which extracts ![](path) syntax from Markdown).
\usepackage{graphicx}
% We will generate all images so they have a width \maxwidth. This means
% that they will get their normal width if they fit onto the page, but
% are scaled down if they would overflow the margins.
\makeatletter
\def\maxwidth{\ifdim\Gin@nat@width>\linewidth\linewidth
	\else\Gin@nat@width\fi}
\makeatother
\let\Oldincludegraphics\includegraphics
% Set max figure width to be 80% of text width, for now hardcoded.
\renewcommand{\includegraphics}[1]{\Oldincludegraphics[width=.8\maxwidth]{#1}}
% Ensure that by default, figures have no caption (until we provide a
% proper Figure object with a Caption API and a way to capture that
% in the conversion process - todo).
\usepackage{caption}
\DeclareCaptionLabelFormat{nolabel}{}
\captionsetup{labelformat=nolabel}

\usepackage{adjustbox} % Used to constrain images to a maximum size 
\usepackage{xcolor} % Allow colors to be defined
\usepackage{enumerate} % Needed for markdown enumerations to work
%\usepackage{geometry} % Used to adjust the document margins
\usepackage{amsmath} % Equations
\usepackage{amssymb} % Equations
\usepackage{textcomp} % defines textquotesingle
% Hack from http://tex.stackexchange.com/a/47451/13684:
\AtBeginDocument{%
	\def\PYZsq{\textquotesingle}% Upright quotes in Pygmentized code
}
\usepackage{upquote} % Upright quotes for verbatim code
\usepackage{eurosym} % defines \euro
\usepackage[mathletters]{ucs} % Extended unicode (utf-8) support
%\usepackage[utf8x]{inputenc} % Allow utf-8 characters in the tex document
\usepackage{fancyvrb} % verbatim replacement that allows latex
\usepackage{grffile} % extends the file name processing of package graphics 
% to support a larger range 
% The hyperref package gives us a pdf with properly built
% internal navigation ('pdf bookmarks' for the table of contents,
% internal cross-reference links, web links for URLs, etc.)
\usepackage{hyperref}
\usepackage{longtable} % longtable support required by pandoc >1.10
\usepackage{booktabs}  % table support for pandoc > 1.12.2
\usepackage[inline]{enumitem} % IRkernel/repr support (it uses the enumerate* environment)
\usepackage[normalem]{ulem} % ulem is needed to support strikethroughs (\sout)
% normalem makes italics be italics, not underlines




% Colors for the hyperref package
\definecolor{urlcolor}{rgb}{0,.145,.698}
\definecolor{linkcolor}{rgb}{.71,0.21,0.01}
\definecolor{citecolor}{rgb}{.12,.54,.11}

% ANSI colors
\definecolor{ansi-black}{HTML}{3E424D}
\definecolor{ansi-black-intense}{HTML}{282C36}
\definecolor{ansi-red}{HTML}{E75C58}
\definecolor{ansi-red-intense}{HTML}{B22B31}
\definecolor{ansi-green}{HTML}{00A250}
\definecolor{ansi-green-intense}{HTML}{007427}
\definecolor{ansi-yellow}{HTML}{DDB62B}
\definecolor{ansi-yellow-intense}{HTML}{B27D12}
\definecolor{ansi-blue}{HTML}{208FFB}
\definecolor{ansi-blue-intense}{HTML}{0065CA}
\definecolor{ansi-magenta}{HTML}{D160C4}
\definecolor{ansi-magenta-intense}{HTML}{A03196}
\definecolor{ansi-cyan}{HTML}{60C6C8}
\definecolor{ansi-cyan-intense}{HTML}{258F8F}
\definecolor{ansi-white}{HTML}{C5C1B4}
\definecolor{ansi-white-intense}{HTML}{A1A6B2}

% commands and environments needed by pandoc snippets
% extracted from the output of `pandoc -s`
\providecommand{\tightlist}{%
	\setlength{\itemsep}{0pt}\setlength{\parskip}{0pt}}
\DefineVerbatimEnvironment{Highlighting}{Verbatim}{commandchars=\\\{\}}
% Add ',fontsize=\small' for more characters per line
\newenvironment{Shaded}{}{}
\newcommand{\KeywordTok}[1]{\textcolor[rgb]{0.00,0.44,0.13}{\textbf{{#1}}}}
\newcommand{\DataTypeTok}[1]{\textcolor[rgb]{0.56,0.13,0.00}{{#1}}}
\newcommand{\DecValTok}[1]{\textcolor[rgb]{0.25,0.63,0.44}{{#1}}}
\newcommand{\BaseNTok}[1]{\textcolor[rgb]{0.25,0.63,0.44}{{#1}}}
\newcommand{\FloatTok}[1]{\textcolor[rgb]{0.25,0.63,0.44}{{#1}}}
\newcommand{\CharTok}[1]{\textcolor[rgb]{0.25,0.44,0.63}{{#1}}}
\newcommand{\StringTok}[1]{\textcolor[rgb]{0.25,0.44,0.63}{{#1}}}
\newcommand{\CommentTok}[1]{\textcolor[rgb]{0.38,0.63,0.69}{\textit{{#1}}}}
\newcommand{\OtherTok}[1]{\textcolor[rgb]{0.00,0.44,0.13}{{#1}}}
\newcommand{\AlertTok}[1]{\textcolor[rgb]{1.00,0.00,0.00}{\textbf{{#1}}}}
\newcommand{\FunctionTok}[1]{\textcolor[rgb]{0.02,0.16,0.49}{{#1}}}
\newcommand{\RegionMarkerTok}[1]{{#1}}
\newcommand{\ErrorTok}[1]{\textcolor[rgb]{1.00,0.00,0.00}{\textbf{{#1}}}}
\newcommand{\NormalTok}[1]{{#1}}

% Additional commands for more recent versions of Pandoc
\newcommand{\ConstantTok}[1]{\textcolor[rgb]{0.53,0.00,0.00}{{#1}}}
\newcommand{\SpecialCharTok}[1]{\textcolor[rgb]{0.25,0.44,0.63}{{#1}}}
\newcommand{\VerbatimStringTok}[1]{\textcolor[rgb]{0.25,0.44,0.63}{{#1}}}
\newcommand{\SpecialStringTok}[1]{\textcolor[rgb]{0.73,0.40,0.53}{{#1}}}
\newcommand{\ImportTok}[1]{{#1}}
\newcommand{\DocumentationTok}[1]{\textcolor[rgb]{0.73,0.13,0.13}{\textit{{#1}}}}
\newcommand{\AnnotationTok}[1]{\textcolor[rgb]{0.38,0.63,0.69}{\textbf{\textit{{#1}}}}}
\newcommand{\CommentVarTok}[1]{\textcolor[rgb]{0.38,0.63,0.69}{\textbf{\textit{{#1}}}}}
\newcommand{\VariableTok}[1]{\textcolor[rgb]{0.10,0.09,0.49}{{#1}}}
\newcommand{\ControlFlowTok}[1]{\textcolor[rgb]{0.00,0.44,0.13}{\textbf{{#1}}}}
\newcommand{\OperatorTok}[1]{\textcolor[rgb]{0.40,0.40,0.40}{{#1}}}
\newcommand{\BuiltInTok}[1]{{#1}}
\newcommand{\ExtensionTok}[1]{{#1}}
\newcommand{\PreprocessorTok}[1]{\textcolor[rgb]{0.74,0.48,0.00}{{#1}}}
\newcommand{\AttributeTok}[1]{\textcolor[rgb]{0.49,0.56,0.16}{{#1}}}
\newcommand{\InformationTok}[1]{\textcolor[rgb]{0.38,0.63,0.69}{\textbf{\textit{{#1}}}}}
\newcommand{\WarningTok}[1]{\textcolor[rgb]{0.38,0.63,0.69}{\textbf{\textit{{#1}}}}}


% Define a nice break command that doesn't care if a line doesn't already
% exist.
\def\br{\hspace*{\fill} \\* }
% Math Jax compatability definitions
\def\gt{>}
\def\lt{<}
% Document parameters
\title{Task 1}




% Pygments definitions

\makeatletter
\def\PY@reset{\let\PY@it=\relax \let\PY@bf=\relax%
	\let\PY@ul=\relax \let\PY@tc=\relax%
	\let\PY@bc=\relax \let\PY@ff=\relax}
\def\PY@tok#1{\csname PY@tok@#1\endcsname}
\def\PY@toks#1+{\ifx\relax#1\empty\else%
	\PY@tok{#1}\expandafter\PY@toks\fi}
\def\PY@do#1{\PY@bc{\PY@tc{\PY@ul{%
				\PY@it{\PY@bf{\PY@ff{#1}}}}}}}
\def\PY#1#2{\PY@reset\PY@toks#1+\relax+\PY@do{#2}}

\expandafter\def\csname PY@tok@gd\endcsname{\def\PY@tc##1{\textcolor[rgb]{0.63,0.00,0.00}{##1}}}
\expandafter\def\csname PY@tok@gu\endcsname{\let\PY@bf=\textbf\def\PY@tc##1{\textcolor[rgb]{0.50,0.00,0.50}{##1}}}
\expandafter\def\csname PY@tok@gt\endcsname{\def\PY@tc##1{\textcolor[rgb]{0.00,0.27,0.87}{##1}}}
\expandafter\def\csname PY@tok@gs\endcsname{\let\PY@bf=\textbf}
\expandafter\def\csname PY@tok@gr\endcsname{\def\PY@tc##1{\textcolor[rgb]{1.00,0.00,0.00}{##1}}}
\expandafter\def\csname PY@tok@cm\endcsname{\let\PY@it=\textit\def\PY@tc##1{\textcolor[rgb]{0.25,0.50,0.50}{##1}}}
\expandafter\def\csname PY@tok@vg\endcsname{\def\PY@tc##1{\textcolor[rgb]{0.10,0.09,0.49}{##1}}}
\expandafter\def\csname PY@tok@vi\endcsname{\def\PY@tc##1{\textcolor[rgb]{0.10,0.09,0.49}{##1}}}
\expandafter\def\csname PY@tok@vm\endcsname{\def\PY@tc##1{\textcolor[rgb]{0.10,0.09,0.49}{##1}}}
\expandafter\def\csname PY@tok@mh\endcsname{\def\PY@tc##1{\textcolor[rgb]{0.40,0.40,0.40}{##1}}}
\expandafter\def\csname PY@tok@cs\endcsname{\let\PY@it=\textit\def\PY@tc##1{\textcolor[rgb]{0.25,0.50,0.50}{##1}}}
\expandafter\def\csname PY@tok@ge\endcsname{\let\PY@it=\textit}
\expandafter\def\csname PY@tok@vc\endcsname{\def\PY@tc##1{\textcolor[rgb]{0.10,0.09,0.49}{##1}}}
\expandafter\def\csname PY@tok@il\endcsname{\def\PY@tc##1{\textcolor[rgb]{0.40,0.40,0.40}{##1}}}
\expandafter\def\csname PY@tok@go\endcsname{\def\PY@tc##1{\textcolor[rgb]{0.53,0.53,0.53}{##1}}}
\expandafter\def\csname PY@tok@cp\endcsname{\def\PY@tc##1{\textcolor[rgb]{0.74,0.48,0.00}{##1}}}
\expandafter\def\csname PY@tok@gi\endcsname{\def\PY@tc##1{\textcolor[rgb]{0.00,0.63,0.00}{##1}}}
\expandafter\def\csname PY@tok@gh\endcsname{\let\PY@bf=\textbf\def\PY@tc##1{\textcolor[rgb]{0.00,0.00,0.50}{##1}}}
\expandafter\def\csname PY@tok@ni\endcsname{\let\PY@bf=\textbf\def\PY@tc##1{\textcolor[rgb]{0.60,0.60,0.60}{##1}}}
\expandafter\def\csname PY@tok@nl\endcsname{\def\PY@tc##1{\textcolor[rgb]{0.63,0.63,0.00}{##1}}}
\expandafter\def\csname PY@tok@nn\endcsname{\let\PY@bf=\textbf\def\PY@tc##1{\textcolor[rgb]{0.00,0.00,1.00}{##1}}}
\expandafter\def\csname PY@tok@no\endcsname{\def\PY@tc##1{\textcolor[rgb]{0.53,0.00,0.00}{##1}}}
\expandafter\def\csname PY@tok@na\endcsname{\def\PY@tc##1{\textcolor[rgb]{0.49,0.56,0.16}{##1}}}
\expandafter\def\csname PY@tok@nb\endcsname{\def\PY@tc##1{\textcolor[rgb]{0.00,0.50,0.00}{##1}}}
\expandafter\def\csname PY@tok@nc\endcsname{\let\PY@bf=\textbf\def\PY@tc##1{\textcolor[rgb]{0.00,0.00,1.00}{##1}}}
\expandafter\def\csname PY@tok@nd\endcsname{\def\PY@tc##1{\textcolor[rgb]{0.67,0.13,1.00}{##1}}}
\expandafter\def\csname PY@tok@ne\endcsname{\let\PY@bf=\textbf\def\PY@tc##1{\textcolor[rgb]{0.82,0.25,0.23}{##1}}}
\expandafter\def\csname PY@tok@nf\endcsname{\def\PY@tc##1{\textcolor[rgb]{0.00,0.00,1.00}{##1}}}
\expandafter\def\csname PY@tok@si\endcsname{\let\PY@bf=\textbf\def\PY@tc##1{\textcolor[rgb]{0.73,0.40,0.53}{##1}}}
\expandafter\def\csname PY@tok@s2\endcsname{\def\PY@tc##1{\textcolor[rgb]{0.73,0.13,0.13}{##1}}}
\expandafter\def\csname PY@tok@nt\endcsname{\let\PY@bf=\textbf\def\PY@tc##1{\textcolor[rgb]{0.00,0.50,0.00}{##1}}}
\expandafter\def\csname PY@tok@nv\endcsname{\def\PY@tc##1{\textcolor[rgb]{0.10,0.09,0.49}{##1}}}
\expandafter\def\csname PY@tok@s1\endcsname{\def\PY@tc##1{\textcolor[rgb]{0.73,0.13,0.13}{##1}}}
\expandafter\def\csname PY@tok@dl\endcsname{\def\PY@tc##1{\textcolor[rgb]{0.73,0.13,0.13}{##1}}}
\expandafter\def\csname PY@tok@ch\endcsname{\let\PY@it=\textit\def\PY@tc##1{\textcolor[rgb]{0.25,0.50,0.50}{##1}}}
\expandafter\def\csname PY@tok@m\endcsname{\def\PY@tc##1{\textcolor[rgb]{0.40,0.40,0.40}{##1}}}
\expandafter\def\csname PY@tok@gp\endcsname{\let\PY@bf=\textbf\def\PY@tc##1{\textcolor[rgb]{0.00,0.00,0.50}{##1}}}
\expandafter\def\csname PY@tok@sh\endcsname{\def\PY@tc##1{\textcolor[rgb]{0.73,0.13,0.13}{##1}}}
\expandafter\def\csname PY@tok@ow\endcsname{\let\PY@bf=\textbf\def\PY@tc##1{\textcolor[rgb]{0.67,0.13,1.00}{##1}}}
\expandafter\def\csname PY@tok@sx\endcsname{\def\PY@tc##1{\textcolor[rgb]{0.00,0.50,0.00}{##1}}}
\expandafter\def\csname PY@tok@bp\endcsname{\def\PY@tc##1{\textcolor[rgb]{0.00,0.50,0.00}{##1}}}
\expandafter\def\csname PY@tok@c1\endcsname{\let\PY@it=\textit\def\PY@tc##1{\textcolor[rgb]{0.25,0.50,0.50}{##1}}}
\expandafter\def\csname PY@tok@fm\endcsname{\def\PY@tc##1{\textcolor[rgb]{0.00,0.00,1.00}{##1}}}
\expandafter\def\csname PY@tok@o\endcsname{\def\PY@tc##1{\textcolor[rgb]{0.40,0.40,0.40}{##1}}}
\expandafter\def\csname PY@tok@kc\endcsname{\let\PY@bf=\textbf\def\PY@tc##1{\textcolor[rgb]{0.00,0.50,0.00}{##1}}}
\expandafter\def\csname PY@tok@c\endcsname{\let\PY@it=\textit\def\PY@tc##1{\textcolor[rgb]{0.25,0.50,0.50}{##1}}}
\expandafter\def\csname PY@tok@mf\endcsname{\def\PY@tc##1{\textcolor[rgb]{0.40,0.40,0.40}{##1}}}
\expandafter\def\csname PY@tok@err\endcsname{\def\PY@bc##1{\setlength{\fboxsep}{0pt}\fcolorbox[rgb]{1.00,0.00,0.00}{1,1,1}{\strut ##1}}}
\expandafter\def\csname PY@tok@mb\endcsname{\def\PY@tc##1{\textcolor[rgb]{0.40,0.40,0.40}{##1}}}
\expandafter\def\csname PY@tok@ss\endcsname{\def\PY@tc##1{\textcolor[rgb]{0.10,0.09,0.49}{##1}}}
\expandafter\def\csname PY@tok@sr\endcsname{\def\PY@tc##1{\textcolor[rgb]{0.73,0.40,0.53}{##1}}}
\expandafter\def\csname PY@tok@mo\endcsname{\def\PY@tc##1{\textcolor[rgb]{0.40,0.40,0.40}{##1}}}
\expandafter\def\csname PY@tok@kd\endcsname{\let\PY@bf=\textbf\def\PY@tc##1{\textcolor[rgb]{0.00,0.50,0.00}{##1}}}
\expandafter\def\csname PY@tok@mi\endcsname{\def\PY@tc##1{\textcolor[rgb]{0.40,0.40,0.40}{##1}}}
\expandafter\def\csname PY@tok@kn\endcsname{\let\PY@bf=\textbf\def\PY@tc##1{\textcolor[rgb]{0.00,0.50,0.00}{##1}}}
\expandafter\def\csname PY@tok@cpf\endcsname{\let\PY@it=\textit\def\PY@tc##1{\textcolor[rgb]{0.25,0.50,0.50}{##1}}}
\expandafter\def\csname PY@tok@kr\endcsname{\let\PY@bf=\textbf\def\PY@tc##1{\textcolor[rgb]{0.00,0.50,0.00}{##1}}}
\expandafter\def\csname PY@tok@s\endcsname{\def\PY@tc##1{\textcolor[rgb]{0.73,0.13,0.13}{##1}}}
\expandafter\def\csname PY@tok@kp\endcsname{\def\PY@tc##1{\textcolor[rgb]{0.00,0.50,0.00}{##1}}}
\expandafter\def\csname PY@tok@w\endcsname{\def\PY@tc##1{\textcolor[rgb]{0.73,0.73,0.73}{##1}}}
\expandafter\def\csname PY@tok@kt\endcsname{\def\PY@tc##1{\textcolor[rgb]{0.69,0.00,0.25}{##1}}}
\expandafter\def\csname PY@tok@sc\endcsname{\def\PY@tc##1{\textcolor[rgb]{0.73,0.13,0.13}{##1}}}
\expandafter\def\csname PY@tok@sb\endcsname{\def\PY@tc##1{\textcolor[rgb]{0.73,0.13,0.13}{##1}}}
\expandafter\def\csname PY@tok@sa\endcsname{\def\PY@tc##1{\textcolor[rgb]{0.73,0.13,0.13}{##1}}}
\expandafter\def\csname PY@tok@k\endcsname{\let\PY@bf=\textbf\def\PY@tc##1{\textcolor[rgb]{0.00,0.50,0.00}{##1}}}
\expandafter\def\csname PY@tok@se\endcsname{\let\PY@bf=\textbf\def\PY@tc##1{\textcolor[rgb]{0.73,0.40,0.13}{##1}}}
\expandafter\def\csname PY@tok@sd\endcsname{\let\PY@it=\textit\def\PY@tc##1{\textcolor[rgb]{0.73,0.13,0.13}{##1}}}

\def\PYZbs{\char`\\}
\def\PYZus{\char`\_}
\def\PYZob{\char`\{}
\def\PYZcb{\char`\}}
\def\PYZca{\char`\^}
\def\PYZam{\char`\&}
\def\PYZlt{\char`\<}
\def\PYZgt{\char`\>}
\def\PYZsh{\char`\#}
\def\PYZpc{\char`\%}
\def\PYZdl{\char`\$}
\def\PYZhy{\char`\-}
\def\PYZsq{\char`\'}
\def\PYZdq{\char`\"}
\def\PYZti{\char`\~}
% for compatibility with earlier versions
\def\PYZat{@}
\def\PYZlb{[}
\def\PYZrb{]}
\makeatother


% Exact colors from NB
\definecolor{incolor}{rgb}{0.0, 0.0, 0.5}
\definecolor{outcolor}{rgb}{0.545, 0.0, 0.0}




% Prevent overflowing lines due to hard-to-break entities
\sloppy 
% Setup hyperref package
\hypersetup{
	breaklinks=true,  % so long urls are correctly broken across lines
	colorlinks=true,
	urlcolor=urlcolor,
	linkcolor=linkcolor,
	citecolor=citecolor,
}
% Slightly bigger margins than the latex defaults

%\geometry{verbose,tmargin=1in,bmargin=1in,lmargin=1in,rmargin=1in}


%%% Работа с русским языком % для pdfLatex
\usepackage{cmap}					% поиск в~PDF
\usepackage{mathtext} 				% русские буквы в~фомулах
\usepackage[T2A]{fontenc}			% кодировка
\usepackage[utf8]{inputenc}			% кодировка исходного текста
\usepackage[english,russian]{babel}	% локализация и переносы
\usepackage{indentfirst} 			% отступ 1 абзаца

%%% Работа с русским языком % для XeLatex
%\usepackage[english,russian]{babel}   %% загружает пакет многоязыковой вёрстки
%\usepackage{fontspec}      %% подготавливает загрузку шрифтов Open Type, True Type и др.
%\defaultfontfeatures{Ligatures={TeX},Renderer=Basic}  %% свойства шрифтов по умолчанию
%\setmainfont[Ligatures={TeX,Historic}]{Times New Roman} %% задаёт основной шрифт документа
%\setsansfont{Comic Sans MS}                    %% задаёт шрифт без засечек
%\setmonofont{Courier New}
%\usepackage{indentfirst}
%\frenchspacing

%%% Дополнительная работа с математикой
\usepackage{amsfonts,amssymb,amsthm,mathtools}
\usepackage{amsmath}
\usepackage{icomma} % "Умная" запятая: $0,2$ --- число, $0, 2$ --- перечисление
\usepackage{upgreek}

%% Номера формул
%\mathtoolsset{showonlyrefs=true} % Показывать номера только у тех формул, на которые есть \eqref{} в~тексте.

%%% Страница
\usepackage{extsizes} % Возможность сделать 14-й шрифт

%% Шрифты
\usepackage{euscript}	 % Шрифт Евклид
\usepackage{mathrsfs} % Красивый матшрифт

%% Свои команды
\DeclareMathOperator{\sgn}{\mathop{sgn}} % создание новой конанды \sgn (типо как \sin)
\usepackage{csquotes} % ещё одна штука для цитат
\newcommand{\pd}[2]{\ensuremath{\cfrac{\partial #1}{\partial #2}}} % частная производная
\newcommand{\abs}[1]{\ensuremath{\left|#1\right|}} % модуль
\renewcommand{\phi}{\ensuremath{\varphi}} % греческая фи
\newcommand{\pogk}[1]{\!\left(\cfrac{\sigma_{#1}}{#1}\right)^{\!\!\!2}\!} % для погрешностей

% Ссылки
\usepackage{color} % подключить пакет color
% выбрать цвета
\definecolor{BlueGreen}{RGB}{49,152,255}
\definecolor{Violet}{RGB}{120,80,120}
% назначить цвета при подключении hyperref
%\usepackage[unicode, colorlinks, urlcolor=blue, linkcolor=blue, pagecolor=blue, citecolor=blue]{hyperref} %синие ссылки
%\usepackage[unicode, colorlinks, urlcolor=black, linkcolor=black, pagecolor=black, citecolor=black]{hyperref} % для печати (отключить верхний!)


%% Перенос знаков в~формулах (по Львовскому)
\newcommand*{\hm}[1]{#1\nobreak\discretionary{}
	{\hbox{$\mathsurround=0pt #1$}}{}}

%%% Работа с картинками
\usepackage{graphicx}  % Для вставки рисунков
\graphicspath{{images/}{images2/}}  % папки с картинками
\setlength\fboxsep{3pt} % Отступ рамки \fbox{} от рисунка
\setlength\fboxrule{1pt} % Толщина линий рамки \fbox{}
\usepackage{wrapfig} % Обтекание рисунков и таблиц текстом
\usepackage{multicol}

%%% Работа с таблицами
\usepackage{array,tabularx,tabulary,booktabs} % Дополнительная работа с таблицами
\usepackage{longtable}  % Длинные таблицы
\usepackage{multirow} % Слияние строк в~таблице
\usepackage{caption}
\captionsetup{labelsep=period, labelfont=bf}

%%% Оформление
\usepackage{indentfirst} % Красная строка
%\setlength{\parskip}{0.3cm} % отступы между абзацами
%%% Название разделов
\usepackage{titlesec}
\titlelabel{\thetitle.\quad}
\renewcommand{\figurename}{\textbf{Рис.}}		%Чтобы вместо figure под рисунками писал "рис"
\renewcommand{\tablename}{\textbf{Таблица}}		%Чтобы вместо table над таблицами писал Таблица

%%% Теоремы
\theoremstyle{plain} % Это стиль по умолчанию, его можно не переопределять.
\newtheorem{theorem}{Теорема}[section]
\newtheorem{proposition}[theorem]{Утверждение}

\theoremstyle{definition} % "Определение"
\newtheorem{definition}{Определение}[section]
\newtheorem{corollary}{Следствие}[theorem]
\newtheorem{problem}{Задача}[section]

\theoremstyle{remark} % "Примечание"
\newtheorem*{nonum}{Решение}
\newtheorem{zamech}{Замечание}[theorem]

%%% Правильные мат. символы для русского языка
\renewcommand{\epsilon}{\ensuremath{\varepsilon}}
\renewcommand{\phi}{\ensuremath{\varphi}}
\renewcommand{\kappa}{\ensuremath{\varkappa}}
\renewcommand{\le}{\ensuremath{\leqslant}}
\renewcommand{\leq}{\ensuremath{\leqslant}}
\renewcommand{\ge}{\ensuremath{\geqslant}}
\renewcommand{\geq}{\ensuremath{\geqslant}}
\renewcommand{\emptyset}{\varnothing}

\usepackage{bm} %жирный греческий шрифт
%\usepackage{ulem}

\graphicspath{{images}}
\usepackage[left=1.27cm,right=1.27cm,top=2cm,bottom=2cm]{geometry}
    

    \begin{document}
    
    
    \maketitle
    
    

    
    \section{Задача 1: сравнение
предложений}\label{ux437ux430ux434ux430ux447ux430-1-ux441ux440ux430ux432ux43dux435ux43dux438ux435-ux43fux440ux435ux434ux43bux43eux436ux435ux43dux438ux439}

    Дан набор предложений, скопированных с Википедии. Каждое из них имеет
"кошачью тему" в одном из трех смыслов:

кошки (животные)

UNIX-утилита cat для вывода содержимого файлов

версии операционной системы OS X, названные в честь семейства кошачьих

Ваша задача --- найти два предложения, которые ближе всего по смыслу к
расположенному в самой первой строке. В качестве меры близости по смыслу
мы будем использовать косинусное расстояние.

    Выполните следующие шаги:

\begin{enumerate}
\def\labelenumi{\arabic{enumi}.}
\tightlist
\item
  Скачайте файл с предложениями (sentences.txt).
\item
  Каждая строка в файле соответствует одному предложению. Считайте их,
  приведите каждую к нижнему регистру с помощью строковой функции
  lower().
\item
  Произведите токенизацию, то есть разбиение текстов на слова. Для этого
  можно воспользоваться регулярным выражением, которое считает
  разделителем любой символ, не являющийся буквой:
  re.split('{[}\^{}a-z{]}', t). Не забудьте удалить пустые слова после
  разделения.
\item
  Составьте список всех слов, встречающихся в предложениях. Сопоставьте
  каждому слову индекс от нуля до (d - 1), где d --- число различных
  слов в предложениях. Для этого удобно воспользоваться структурой dict.
\item
  Создайте матрицу размера n * d, где n --- число предложений. Заполните
  ее: элемент с индексом (i, j) в этой матрице должен быть равен
  количеству вхождений j-го слова в i-е предложение. У вас должна
  получиться матрица размера 22 * 254.
\item
  Найдите косинусное расстояние от предложения в самой первой строке (In
  comparison to dogs, cats have not undergone...) до всех остальных с
  помощью функции scipy.spatial.distance.cosine. Какие номера у двух
  предложений, ближайших к нему по этому расстоянию (строки нумеруются с
  нуля)? Эти два числа и будут ответами на задание. Само предложение (In
  comparison to dogs, cats have not undergone... ) имеет индекс 0.
\item
  Запишите полученные числа в файл, разделив пробелом. Обратите
  внимание, что файл должен состоять из одной строки, в конце которой не
  должно быть переноса. Пример файла с решением вы можете найти в конце
  задания (submission-1.txt).
\item
  Совпадают ли ближайшие два предложения по тематике с первым? Совпадают
  ли тематики у следующих по близости предложений?
\end{enumerate}

Разумеется, использованный вами метод крайне простой. Например, он не
учитывает формы слов (так, cat и cats он считает разными словами, хотя
по сути они означают одно и то же), не удаляет из текстов артикли и
прочие ненужные слова. Позже мы будем подробно изучать анализ текстов,
где выясним, как достичь высокого качества в задаче поиска похожих
предложений.

    \begin{Verbatim}[commandchars=\\\{\}]
{\color{incolor}In [{\color{incolor}12}]:} \PY{k+kn}{import} \PY{n+nn}{re}
         \PY{k+kn}{import} \PY{n+nn}{numpy} \PY{k+kn}{as} \PY{n+nn}{np}
         \PY{k+kn}{import} \PY{n+nn}{scipy}
         
         \PY{n}{s\PYZus{}file} \PY{o}{=} \PY{n+nb}{open}\PY{p}{(}\PY{l+s+s1}{\PYZsq{}}\PY{l+s+s1}{sentences.txt}\PY{l+s+s1}{\PYZsq{}}\PY{p}{)}
         \PY{n}{n} \PY{o}{=} \PY{l+m+mi}{0}
         \PY{k}{for} \PY{n}{line} \PY{o+ow}{in} \PY{n}{s\PYZus{}file}\PY{p}{:}
             \PY{n}{n} \PY{o}{+}\PY{o}{=} \PY{l+m+mi}{1}
         \PY{n}{s\PYZus{}file} \PY{o}{=} \PY{n+nb}{open}\PY{p}{(}\PY{l+s+s1}{\PYZsq{}}\PY{l+s+s1}{sentences.txt}\PY{l+s+s1}{\PYZsq{}}\PY{p}{)}
         \PY{n}{sentences} \PY{o}{=} \PY{n}{s\PYZus{}file}\PY{o}{.}\PY{n}{read}\PY{p}{(}\PY{p}{)}\PY{o}{.}\PY{n}{lower}\PY{p}{(}\PY{p}{)}
         \PY{n}{s\PYZus{}file}\PY{o}{.}\PY{n}{close}\PY{p}{(}\PY{p}{)}
\end{Verbatim}


    \begin{Verbatim}[commandchars=\\\{\}]
{\color{incolor}In [{\color{incolor}13}]:} \PY{n}{sentences} \PY{o}{=} \PY{n}{re}\PY{o}{.}\PY{n}{findall}\PY{p}{(}\PY{l+s+sa}{r}\PY{l+s+s1}{\PYZsq{}}\PY{l+s+s1}{[a\PYZhy{}z]+}\PY{l+s+s1}{\PYZsq{}}\PY{p}{,} \PY{n}{sentences}\PY{p}{)}
\end{Verbatim}


    \begin{Verbatim}[commandchars=\\\{\}]
{\color{incolor}In [{\color{incolor}14}]:} \PY{n}{words} \PY{o}{=} \PY{p}{\PYZob{}}\PY{p}{\PYZcb{}}
         \PY{k}{for} \PY{n}{word} \PY{o+ow}{in} \PY{n}{sentences}\PY{p}{:}
             \PY{k}{if} \PY{n}{word} \PY{o+ow}{in} \PY{n}{words}\PY{p}{:}
                 \PY{n}{words}\PY{p}{[}\PY{n}{word}\PY{p}{]} \PY{o}{+}\PY{o}{=} \PY{l+m+mi}{1}
             \PY{k}{else}\PY{p}{:}
                 \PY{n}{words}\PY{p}{[}\PY{n}{word}\PY{p}{]} \PY{o}{=} \PY{l+m+mi}{1}
         
         \PY{n}{count} \PY{o}{=} \PY{p}{\PYZob{}}\PY{p}{\PYZcb{}}
         \PY{n}{i} \PY{o}{=} \PY{l+m+mi}{0}
         \PY{k}{for} \PY{n}{word} \PY{o+ow}{in} \PY{n}{words}\PY{p}{:}
             \PY{n}{count}\PY{p}{[}\PY{n}{word}\PY{p}{]} \PY{o}{=} \PY{n}{i}
             \PY{n}{i} \PY{o}{+}\PY{o}{=} \PY{l+m+mi}{1}
\end{Verbatim}


    \begin{Verbatim}[commandchars=\\\{\}]
{\color{incolor}In [{\color{incolor}15}]:} \PY{n}{d} \PY{o}{=} \PY{n+nb}{len}\PY{p}{(}\PY{n}{words}\PY{p}{)}
\end{Verbatim}


    \begin{Verbatim}[commandchars=\\\{\}]
{\color{incolor}In [{\color{incolor}16}]:} \PY{n}{frec} \PY{o}{=} \PY{n}{np}\PY{o}{.}\PY{n}{zeros}\PY{p}{(}\PY{p}{(}\PY{n}{n}\PY{p}{,} \PY{n}{d}\PY{p}{)}\PY{p}{)}
         
         \PY{n}{s\PYZus{}file} \PY{o}{=} \PY{n+nb}{open}\PY{p}{(}\PY{l+s+s1}{\PYZsq{}}\PY{l+s+s1}{sentences.txt}\PY{l+s+s1}{\PYZsq{}}\PY{p}{)}
         \PY{n}{s} \PY{o}{=} \PY{n}{s\PYZus{}file}\PY{o}{.}\PY{n}{read}\PY{p}{(}\PY{p}{)}\PY{o}{.}\PY{n}{lower}\PY{p}{(}\PY{p}{)}
         
         \PY{n}{i} \PY{o}{=} \PY{l+m+mi}{0}
         \PY{n}{res} \PY{o}{=} \PY{n}{s}\PY{o}{.}\PY{n}{split}\PY{p}{(}\PY{l+s+s1}{\PYZsq{}}\PY{l+s+se}{\PYZbs{}n}\PY{l+s+s1}{\PYZsq{}}\PY{p}{)}
         \PY{k}{for} \PY{n}{line} \PY{o+ow}{in} \PY{n}{res}\PY{p}{:}
             \PY{n}{res}\PY{p}{[}\PY{n}{i}\PY{p}{]} \PY{o}{=} \PY{n}{re}\PY{o}{.}\PY{n}{findall}\PY{p}{(}\PY{l+s+sa}{r}\PY{l+s+s1}{\PYZsq{}}\PY{l+s+s1}{[a\PYZhy{}z]+}\PY{l+s+s1}{\PYZsq{}}\PY{p}{,} \PY{n}{line}\PY{p}{)}
             \PY{n}{i} \PY{o}{+}\PY{o}{=} \PY{l+m+mi}{1}
         
         \PY{n}{i} \PY{o}{=} \PY{l+m+mi}{0}
         \PY{k}{for} \PY{n}{line} \PY{o+ow}{in} \PY{n}{res}\PY{p}{:}
             \PY{k}{for} \PY{n}{word} \PY{o+ow}{in} \PY{n}{line}\PY{p}{:}
                 \PY{n}{frec}\PY{p}{[}\PY{n}{i}\PY{p}{]}\PY{p}{[}\PY{n}{count}\PY{p}{[}\PY{n}{word}\PY{p}{]}\PY{p}{]} \PY{o}{+}\PY{o}{=} \PY{l+m+mi}{1}
             \PY{n}{i} \PY{o}{+}\PY{o}{=} \PY{l+m+mi}{1}
         \PY{k}{print} \PY{n}{frec}\PY{o}{.}\PY{n}{shape}
         \PY{n}{s\PYZus{}file}\PY{o}{.}\PY{n}{close}\PY{p}{(}\PY{p}{)}
\end{Verbatim}


    \begin{Verbatim}[commandchars=\\\{\}]
(22L, 254L)

    \end{Verbatim}

    \begin{Verbatim}[commandchars=\\\{\}]
{\color{incolor}In [{\color{incolor}17}]:} \PY{k+kn}{import} \PY{n+nn}{scipy.spatial}
         \PY{n}{dist} \PY{o}{=} \PY{p}{[}\PY{p}{]}
         \PY{k}{for} \PY{n}{i} \PY{o+ow}{in} \PY{n+nb}{range}\PY{p}{(}\PY{n}{n}\PY{p}{)}\PY{p}{:}
             \PY{n}{dist}\PY{o}{.}\PY{n}{append}\PY{p}{(}\PY{n}{scipy}\PY{o}{.}\PY{n}{spatial}\PY{o}{.}\PY{n}{distance}\PY{o}{.}\PY{n}{cosine}\PY{p}{(}\PY{n}{frec}\PY{p}{[}\PY{l+m+mi}{0}\PY{p}{]}\PY{p}{,} \PY{n}{frec}\PY{p}{[}\PY{n}{i}\PY{p}{]}\PY{p}{)}\PY{p}{)}
\end{Verbatim}


    \begin{Verbatim}[commandchars=\\\{\}]
{\color{incolor}In [{\color{incolor}18}]:} \PY{k}{with} \PY{n+nb}{open}\PY{p}{(}\PY{l+s+s2}{\PYZdq{}}\PY{l+s+s2}{results.txt}\PY{l+s+s2}{\PYZdq{}}\PY{p}{,} \PY{l+s+s1}{\PYZsq{}}\PY{l+s+s1}{w}\PY{l+s+s1}{\PYZsq{}}\PY{p}{)} \PY{k}{as} \PY{n}{file\PYZus{}res}\PY{p}{:}
             \PY{n}{final\PYZus{}result} \PY{o}{=} \PY{l+s+s1}{\PYZsq{}}\PY{l+s+s1}{\PYZsq{}}
             \PY{k}{for} \PY{n}{elem} \PY{o+ow}{in} \PY{n}{dist}\PY{p}{:}
                 \PY{n}{final\PYZus{}result} \PY{o}{+}\PY{o}{=} \PY{n+nb}{str}\PY{p}{(}\PY{n}{elem}\PY{p}{)} \PY{o}{+} \PY{l+s+s2}{\PYZdq{}}\PY{l+s+s2}{ }\PY{l+s+s2}{\PYZdq{}}
             \PY{n}{file\PYZus{}res}\PY{o}{.}\PY{n}{write}\PY{p}{(}\PY{n}{final\PYZus{}result}\PY{p}{)}
\end{Verbatim}


    \begin{Verbatim}[commandchars=\\\{\}]
{\color{incolor}In [{\color{incolor}19}]:} \PY{n}{tmp} \PY{o}{=} \PY{p}{[}\PY{p}{]}
         \PY{n}{tmp}\PY{p}{[}\PY{p}{:}\PY{p}{]} \PY{o}{=} \PY{n}{dist}\PY{p}{[}\PY{p}{:}\PY{p}{]}
         \PY{n}{tmp}\PY{o}{.}\PY{n}{pop}\PY{p}{(}\PY{l+m+mi}{0}\PY{p}{)}
         \PY{n}{ans\PYZus{}1} \PY{o}{=} \PY{n}{dist}\PY{o}{.}\PY{n}{index}\PY{p}{(}\PY{n+nb}{min}\PY{p}{(}\PY{n}{tmp}\PY{p}{)}\PY{p}{)}
         \PY{n}{tmp}\PY{o}{.}\PY{n}{pop}\PY{p}{(}\PY{n}{tmp}\PY{o}{.}\PY{n}{index}\PY{p}{(}\PY{n+nb}{min}\PY{p}{(}\PY{n}{tmp}\PY{p}{)}\PY{p}{)}\PY{p}{)}
         \PY{n}{ans\PYZus{}2} \PY{o}{=} \PY{n}{dist}\PY{o}{.}\PY{n}{index}\PY{p}{(}\PY{n+nb}{min}\PY{p}{(}\PY{n}{tmp}\PY{p}{)}\PY{p}{)}
\end{Verbatim}


    \begin{Verbatim}[commandchars=\\\{\}]
{\color{incolor}In [{\color{incolor}20}]:} \PY{k}{with} \PY{n+nb}{open}\PY{p}{(}\PY{l+s+s2}{\PYZdq{}}\PY{l+s+s2}{submission\PYZhy{}1.txt}\PY{l+s+s2}{\PYZdq{}}\PY{p}{,} \PY{l+s+s1}{\PYZsq{}}\PY{l+s+s1}{w}\PY{l+s+s1}{\PYZsq{}}\PY{p}{)} \PY{k}{as} \PY{n}{file\PYZus{}ans}\PY{p}{:}
             \PY{n}{s} \PY{o}{=} \PY{n+nb}{str}\PY{p}{(}\PY{n}{ans\PYZus{}1}\PY{p}{)} \PY{o}{+} \PY{l+s+s2}{\PYZdq{}}\PY{l+s+s2}{ }\PY{l+s+s2}{\PYZdq{}} \PY{o}{+} \PY{n+nb}{str}\PY{p}{(}\PY{n}{ans\PYZus{}2}\PY{p}{)}
             \PY{n}{file\PYZus{}ans}\PY{o}{.}\PY{n}{write}\PY{p}{(}\PY{n}{s}\PY{p}{)}
         \PY{k}{print} \PY{n}{s}
\end{Verbatim}


    \begin{Verbatim}[commandchars=\\\{\}]
6 4

    \end{Verbatim}


    % Add a bibliography block to the postdoc
    
    
    
    \end{document}
